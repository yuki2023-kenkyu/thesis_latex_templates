\documentclass[luatex,base=12pt]{japanese_thesis}
\usepackage{./preambles/packages_lualatex}
\usepackage{./preambles/macros}
\usepackage{booktabs}
\providecommand{\zcref}[2][]{\ref{#2}}
\providecommand{\zcite}[1]{\cite{#1}} % if you define such a macro in your thesis
\providecommand{\num}[1]{#1}
\providecommand{\unit}[1]{\mathrm{#1}}
\providecommand{\qty}[2]{#1\,\mathrm{#2}}

% 参考文献ファイルの読み込み
\addbibresource{./references/reference_1.bib}
% 表紙の出力情報の設定
% \SetMasterThesis % 修士論文の場合コメントを外す
\SetDoctoralThesis % 博士論文の場合コメントを外す
\Title{チュートリアル} % タイトル
\StudentNumber{123456789} % 学籍番号
\Author{物理 \ 太郎} % 著者名
\Professor{物理 \ 花子} % 指導教員名
\DateofSubmission{6} % 年度

\begin{document}
\Maketitle
\tableofcontents
\mainmatter

\chapter{本ドキュメントの目的と適用範囲}
本 Style Guide は,\texttt{thesis\_latex\_templates}(\texttt{japanese\_thesis.cls})を用いて卒業論文・修士論文を執筆する際に,
文書全体の体裁・表記・数式・図表・参考文献・付録などの運用ルールを統一することを目的とする.
本ガイドラインはあくまで「推奨ルール」を示すものであり,学科・指導教員の規定が優先される.
ただし,特段の理由がない限り,本ガイドラインに従うことを推奨する.

\chapter{推奨ビルド環境}
\begin{itemize}
  \item コンパイル:LuaLaTeX(テンプレート既定)
  \item 参考文献:BibLaTeX + Biber を推奨(テンプレート設定に従う)
\end{itemize}

\chapter{文章表記の基本ルール}
\section{句読点・文体・段落}
\begin{itemize}
  \item 句読点は全角コンマと全角ピリオド(「,.」)に統一する.
  \item 文体は常体(だ・である調)に統一する.
  \item 段落の字下げは,インデント命令ではなく「改行」により管理する(テンプレートの既定に従う).
\end{itemize}

\section{見出しの付け方}
\begin{itemize}
  \item 「目的」「背景」「実験」「結果」など単語だけの章・節名は避け,章の内容が一目で分かる具体的なタイトルにする.
  \item 目次だけを見て論文の流れが追えるように章立てを設計する.
  \item 章数は概ね 5~8 章を目安とする(研究分野・規定に応じて調整).
  \item 各章の冒頭に,当該章の要約(章で何を行い,何が得られるか)を置く.
\end{itemize}

\chapter{論文構成の推奨(卒論・修論 共通)}
\begin{enumerate}
  \item 序論:研究背景・先行研究での位置づけ・目的・方法(全体像)を明確に述べる.
  \item 本論:理論/方法/実験・解析/結果/考察を,読者が再現できる粒度で記述する.
  \item 結論:目的に対して何を行い,何が得られ,何が残課題かを整理する.
  \item 謝辞:結論と参考文献の間に置く(指導教員・査読者・データ提供者など).
  \item 参考文献:本文で引用したもののみを収録し,番号対応を一致させる.
  \item 付録:補足導出,追加図表,使用コード等(必要に応じて).
\end{enumerate}

\chapter{数式(Notation と組版)}
\section{原則}
\begin{itemize}
  \item 記号は一貫性を最優先し,定義した記号は最後まで同じ意味で使う.
  \item ベクトルは太字にする(例:$\vect{x}$).
  \item \textbf{単位・数値は \texttt{siunitx} を用いて表記を統一する}(例:\verb|\qty{70}{\kilogram\meter\per\second}|).
  \item \textbf{相互参照は \texttt{zref-clever} の  を用いて統一する}(式・図・表・節などを同一の作法で参照する).
  \item \textbf{長い数式は \texttt{dmath} 環境(breqn)を優先する}(自動改行と番号付けの一貫性).
  \item 括弧サイズの自動調整は \textbf{\texttt{physics2} を優先}し,\verb|\ab( ... )|(必要に応じて \verb|\pab|, \verb|\Bab| など)を用いる.
  \item 表示数式も文章の一部として句読点を付ける(必要なら数式末尾に「,」「.」を置く).
\end{itemize}

\section{例:文章内参照(\texttt{\textbackslash zcref})}
次の\zcref{eq:example}は例である.
\begin{equation}
  E = \frac{1}{2} m v^2 \, .
  \label{eq:example}
\end{equation}

\section{相互参照(zref-clever / \texttt{\textbackslash zcref})}
本テンプレートでは,本文中で番号を参照する場合は \verb|\zcref| を用いる.
\begin{itemize}
  \item \verb|\zcref{eq:...}|:式参照(例:\verb|\zcref{eq:einstein}|)
  \item \verb|\zcref{fig:...}|:図参照(例:\verb|\zcref{fig:hz}|)
  \item \verb|\zcref{tab:...}|:表参照(例:\verb|\zcref{tab:data}|)
  \item \verb|\zcref{sec:...}|:節参照(例:\verb|\zcref{sec:method}|)
\end{itemize}

\noindent
\textbf{命名規則}:ラベルは \verb|eq:|, \verb|fig:|, \verb|tab:|, \verb|sec:| などの接頭辞を付け,参照対象の種別が一目で分かるようにする.

\medskip
\noindent
\textbf{文頭の参照}:文頭で参照を置く場合は,必要に応じて \verb|\zcref[S]{...}|(先頭大文字化等のローカルオプション)を用いる.

\medskip
\noindent
\textbf{複数参照・範囲圧縮}:\verb|\zcref{eq:a,eq:b,eq:c}| のように複数ラベルを並べると,自動で整形(並べ替え・範囲圧縮)される設定を推奨する.

\section{長い数式(breqn / dmath)}
長い式(右端にはみ出す式)を無理に \texttt{align} や手動改行で処理すると,括弧や番号付けの一貫性が崩れやすい.
本テンプレートでは \texttt{breqn} の \texttt{dmath} 環境を用いる(推奨).

\begin{verbatim}
\begin{dmath}
  S = \int d^4x \sqrt{-g}\,
      \qty[ \frac12 (M_{\mathrm{Pl}}^2-\alpha\varphi^2)R
          + \frac12 A R^2
          - \frac12 g^{\mu\nu}\partial_\mu\varphi\partial_\nu\varphi
          - V(\varphi) ] .
  \label{eq:action}
\end{dmath}

本文では \zcref{eq:action} のように参照する.
\end{verbatim}

\noindent
\textbf{注意}:\texttt{dmath} の有無や設定(\verb|\breqnsetup| 等)はテンプレート側のプリアンブルに依存する.テンプレート既定と異なる挙動が必要な場合は,\texttt{preambles/} 側の設定と整合するように調整する.

\section{単位・数値(siunitx)}
単位は原則 \texttt{siunitx} を用いて表記する.典型例:
\begin{itemize}
  \item 数値のみ:\verb|\num{1.23e-4}|
  \item 単位のみ:\verb|\unit{\kilo\metre\per\second\per\mega\parsec}|
  \item 物理量(数値×単位):\verb|\qty{70}{\kilo\metre\per\second\per\mega\parsec}|
\end{itemize}

\noindent
\textbf{軸ラベル例}(数式中):\verb|$H(z)\,(\unit{\kilo\metre\per\second\per\mega\parsec})$|.

\medskip
\noindent
\textbf{表での桁揃え}:\texttt{siunitx} の \texttt{S} 列型を使うと,小数点位置での整列が可能(必要に応じてテンプレート設定に合わせる).

\section{数式入力の省力化(physics2)}
本テンプレートでは,従来の \texttt{physics} パッケージに代えて \texttt{physics2} を用い,
\texttt{siunitx} と衝突しやすい \verb|\qty| 等を回避しつつ,数式入力を簡潔にする運用を推奨する.

\begin{itemize}
  \item \textbf{自動括弧}:\verb|\ab( ... )|, \verb|\ab[ ... ]|, \verb|\ab\{ ... \}| など
  \item \textbf{派生コマンド}:\verb|\pab{...}|(丸括弧),\verb|\bab{...}|(角括弧),\verb|\Bab{...}|(波括弧)など
  \item 旧 \texttt{physics} 文書の保守が必要な場合は,\texttt{physics2} の \texttt{*.legacy} モジュールで互換コマンドを段階的に導入する
\end{itemize}

\noindent
\textbf{実務上の指針}:
\begin{enumerate}
  \item 新規執筆では \verb|\left|,\verb|\right| の多用は避け,\verb|\ab| を第一選択とする(括弧サイズの一貫性と可読性のため).
  \item 既存原稿で \verb|\qty| を用いている場合は,まず \verb|\qty| を \verb|\ab| に置換し,次に必要に応じて \texttt{ab.legacy} 等の導入を検討する.
\end{enumerate}

\chapter{図(Figure)}
\section{原則}
\begin{itemize}
  \item 図は,研究で明らかにしたい内容に合致し,単体で理解できるキャプションを付ける.
  \item 軸ラベルには「量」と「単位」を入れ,単位は括弧で示す(例:$H(z)$,(\unit{\kilogram\meter\per\second})).
  \item 凡例・記号・線種は判読可能なサイズにし,白黒印刷でも識別できるよう線種も併用する.
  \item 本文では「図\#」でなく参照コマンドで引用し,出現順に番号が振られるようにする.
\end{itemize}

\section{例}
\begin{figure}[tb]
  \centering
  \includegraphics[width=0.7\linewidth]{./images/論文様式.png}
  \caption{例:結果を要約する図.曲線の意味や条件は本文を見なくても分かるように書く.}
  \label{fig:example}
\end{figure}

本文では \zcref{fig:example} として参照する.

\chapter{表(Table)}
\section{原則(PRD/APS流の推奨を採用)}
\begin{itemize}
  \item 表には内容が自明になるキャプションを付け,記号の定義や単位は見出し・キャプションに含める.
  \item 列見出しに単位をまとめ,縦罫線は原則使わない.水平罫線も最小限(\texttt{booktabs})とする.
\end{itemize}

\section{例(booktabs)}
\begin{table}[tb]
  \caption{例:主要パラメータ(単位は見出しに含める).}
  \label{tab:params}
  \centering
  \begin{tabular}{lcc}
    \toprule
    Parameter & Value & Unit \\
    \midrule
    $H_0$ & \num{70} & \unit{\kilogram\meter\per\second} \\
    $\Omega_{m0}$ & 0.3 & --- \\
    \bottomrule
  \end{tabular}
\end{table}

\chapter{参考文献(Citations)}
\begin{itemize}
  \item 引用した文献はすべて参考文献リストに含め,逆に本文で引用していない文献はリストに入れない.
  \item 査読付き学術雑誌論文など「一次文献」を主とする(可能な限り).
  \item 本文中の引用形式(番号/著者年)と参考文献リストの形式はテンプレート設定に合わせて統一する.
\end{itemize}

\section{BibLaTeX/Biber の基本運用}
本テンプレートでは \texttt{.bib} ファイルを用いて文献管理する.原則として,
\begin{itemize}
  \item プリアンブルで \verb|\addbibresource{reference.bib}| を宣言する(ファイル名は運用に合わせて変更可).
  \item 本文中は \verb|\cite{key}|(またはテンプレートで定義したコマンド)で引用する.
  \item 末尾に \verb|\printbibliography| を置く.
\end{itemize}
\noindent
コンパイル手順は(例)LuaLaTeX $\rightarrow$ Biber $\rightarrow$ LuaLaTeX $\rightarrow$ LuaLaTeX とし,\texttt{latexmk} を推奨する.

\section{日本語文献を .bib に登録する際の注意(重要)}
日本語文献を混在させる場合は,次を推奨する.
\begin{itemize}
  \item \textbf{UTF-8} で \texttt{.bib} を保存する(LuaLaTeX + Biber を前提).
  \item \textbf{\texttt{langid = \{japanese\}}} を付与する(言語別整形・表示の補助).
  \item ソート(並び順)が崩れる場合は,\texttt{sortname} などでローマ字の並びキーを明示する(例:\verb|sortname = {Yamada, Taro}|).
  \item 著者名の分割を避けたい場合は,\texttt{author = \{\{山田 太郎\}\}} のように二重波括弧で保護する(スタイル依存).
  \item 日本語タイトル等で大文字小文字保持が必要な場合は,必要箇所を \verb|{...}| で保護する(英語文献の固有名詞等).
\end{itemize}

\section{日本語文献の登録例(BibLaTeX)}
\begin{verbatim}
@book{Yamada2020GR,
  author    = {{山田 太郎}},
  title     = {{一般相対論入門}},
  date      = {2020},
  publisher = {架空出版社},
  location  = {東京},
  langid    = {japanese},
  sortname  = {Yamada, Taro},
}

@article{Suzuki2019Cosmo,
  author       = {{鈴木 花子}},
  title        = {{初期宇宙論の基礎}},
  journaltitle = {{日本物理学会誌}},
  date         = {2019},
  volume       = {74},
  number       = {3},
  pages        = {123-130},
  langid       = {japanese},
  sortname     = {Suzuki, Hanako},
}
\end{verbatim}

\chapter{データ・コード(再現性)}
\begin{itemize}
  \item 外部ソースのコード(例:GitHub)を使用した場合は出典 URL を明記する.
  \item 数値解析の場合,対象方程式・初期条件・パラメータ設定・数値解法(アルゴリズム)を明記する.
  \item 使用コードは付録にまとめる(またはリポジトリに収録し,参照先を明記する).
\end{itemize}

\chapter{付録(Appendix)}
付録を用いる場合,章番号の出し方(A, B, \dots / 付録A など)は学科規定とテンプレート設定に従う.
テンプレートの既定が「章番号付きの付録章」になる場合は,規定に応じて \verb|\chapter*| 等で番号を抑制する.

\chapter{提出前チェックリスト(抜粋)}
\begin{itemize}
  \item 句読点は「,.」に統一されている.
  \item 文体(常体)は統一されている.
  \item 目次ページ番号がローマン体になっている(テンプレート設定を確認).
  \item 各章冒頭に要約がある.章数は概ね 5~8 章の範囲に収まっている.
  \item 参照は \verb|\zcref| 等で統一され(式・図・表・節),ベクトル表記は太字になっている.
  \item 図表は出現順に参照され,キャプションが単体で理解できる.
  \item 謝辞が「結論と参考文献の間」にある.
  \item 参考文献は本文で引用したもののみが列挙され,番号対応が一致している.
  \item 外部コード・データの出典が明記されている.
\end{itemize}

\end{document}