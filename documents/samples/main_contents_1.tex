\chapter{\LaTeX の基本的な使い方}

    \section{\LaTeX 文書の構造}
    \LaTeX ドキュメントは,以下のような構造で記述します.
        \begin{verbatim}
            \documentclass{jbook}
            \usepackage{amsmath, amssymb}
            \usepackage{graphicx}
            \usepackage{geometry}
            \usepackage{biblatex}
            \usepackage{physics}
            \usepackage{hyperref}
            \usepackage{cleveref}
            \addbibresource{reference.bib}
            % 自作マクロの定義
            \newcommand{\Planck}{h}
            \newcommand{\myfrac}[2]{\frac{#1}{#2}}
            \newcommand{\veca}[1]{\vec{#1}}
            \newcommand{\diff}[2]{\frac{d#1}{d#2}}
            \newcommand{\cLight}{\ensuremath{c}}
            \newcommand{\simplefrac}[2][1]{\frac{#2}{#1}}

            % 新しい数学演算子の定義
            \DeclareMathOperator{\tr}{tr}

            \begin{document}
                \title{卒業論文のタイトル}
                \author{氏名}
                \date{提出日}

                \maketitle

                \tableofcontents

                \chapter{序論}
                ここに序論を書く.

                \chapter{理論背景}
                ここに理論背景を書く.

                \chapter{実験方法}
                ここに実験方法を書く.

                \chapter{結果と考察}
                ここに結果と考察を書く.

                \chapter{結論}
                ここに結論を書く.

                \printbibliography

            \end{document}
        \end{verbatim}

    \section{便利なパッケージの紹介}
        \subsection{physics2}
        物理系の数式を簡略に記述できるパッケージです.たとえば,次のように記述が簡単になります.

            \begin{verbatim}
            \ab() % 括弧()の大きさ自動調整
            \end{verbatim}

        ただし,従来のPhysicsパッケージとは異なり,\verb|\qty{}|コマンドを始めとしたコマンドが使えないため,他の数学パッケージの機能を複数組み合わせて書く必要があります.

        また,Physics2パッケージにおいて括弧の大きさの自動調整機能を利用するには,\verb|\usephysicsmodule{ab}|をプレアンブルに追加する必要があります.

        \subsection{zref-clever}
        zref-clever パッケージは,参照コマンド(\verb|ref{}|)を強化し,ラベルの種類(図,表,方程式,ソースコード,章など)に応じた適切な表現を自動的に行います.

            \begin{verbatim}
            \zcref{fig:example} % 出力: Fig.1
            \end{verbatim}

    \section{数式入力の方法}
        数式はインラインモード(\verb|$...$|)またはディスプレイモード(\verb|$$...$$| または \verb|\begin{equation}...\end{equation}|)で記述します.

        \subsection{インライン数式}
            \begin{verbatim}
                インライン数式: $E = mc^2$
            \end{verbatim}

        \subsection{ディスプレイ数式}
            \begin{verbatim}
                \begin{equation}
                    E = mc^2
                \end{equation}
            \end{verbatim}

    \section{図や表の挿入方法}
        \subsection{図の挿入}
        以下のコードで図を挿入できます.
            \begin{verbatim}
                \begin{figure}[h]
                    \centering
                    \includegraphics[width=0.8\textwidth]{figure.png}
                    \caption{サンプル図}
                    \label{fig:example}
                \end{figure}
            \end{verbatim}

        \subsection{表の挿入}
        以下のコードで表を挿入します.
            \begin{verbatim}
                \begin{table}[h]
                    \centering
                    \begin{tabular}{|c|c|c|}
                    \hline
                    列1 & 列2 & 列3 \\
                    \hline
                    データ1 & データ2 & データ3 \\
                    \hline
                    \end{tabular}
                    \caption{サンプル表}
                    \label{tab:example}
                \end{table}
            \end{verbatim}

    \section{参考文献の管理(biblatex)}
        \subsection{文献ファイルの準備}
        .bibファイルに文献データを記述します.
        文献情報は,例えば次のように記述します.
        \begin{verbatim}
            @article{Einstein1905,%引用するときのラベル
                author = {Einstein, Albert},
                title = {Zur Elektrodynamik bewegter K{\"o}rper},
                journal = {Annalen der Physik},
                volume = {322},
                number = {10},
                pages = {891--921},
                year = {1905},
                publisher = {Wiley Online Library},
                doi = {10.1002/andp.19053221004}
            }
        \end{verbatim}

        \subsection{論文の引用方法}
        次のコマンドを用いて論文を文中で引用します.
            \begin{verbatim}
                \cite{引用したい論文のラベル} %論文の引用
            \end{verbatim}

        \subsection{参考文献リストの表示}
        次のコマンドを用いて参考文献リストを表示します.
            \begin{verbatim}
                \printbibliography %参考文献リストの表示
            \end{verbatim}

    \section{マクロ定義方法}
        \subsection{基本的なマクロの定義}
            \begin{verbatim}
                \newcommand{\Planck}{h}
                \newcommand{\myfrac}[2]{\frac{#1}{#2}}
                \newcommand{\veca}[1]{\vec{#1}}
                \newcommand{\diff}[2]{\frac{d#1}{d#2}}
                \newcommand{\cLight}{\ensuremath{c}}
                \newcommand{\simplefrac}[2][1]{\frac{#2}{#1}}
            \end{verbatim}

        \subsection{オプション引数の設定}
            \begin{verbatim}
                \newcommand{\simplefrac}[2][1]{\frac{#2}{#1}}
            \end{verbatim}

        \subsection{既存コマンドの再定義と数学演算子の定義}
            \begin{verbatim}
                \renewcommand{\vec}[1]{\mathbf{#1}}
                \DeclareMathOperator{\tr}{tr}
            \end{verbatim}

