\section{非特異バウンス宇宙モデル}
    \subsection{作用}

    時空の線素はフリードマン・ルメートル・ロバートソン・ウォーカー(FLRW)計量で記述される一様等方背景のもと次の形で与えられる。
        \begin{dmath}
        ds^2 = -dt^2 + a^2(t) \left( \frac{dr^2}{1 - kr^2} + r^2 d\theta^2 + r^2 \sin^2 \theta d\phi^2 \right)
        \end{dmath}
        ここで,$a(t)$は宇宙の大きさを表すスケール因子,$k$は空間曲率である。また,本研究では,スカラー場$\phi$とスカラー曲率$R$が非最小結合したStarobinsky項($R^2$項)を含む次の作用を考える。

        \begin{dmath}
        S = \int d^4x \sqrt{-g} \left[ \frac{1}{2}(M_{\text{Pl}}^2 - \alpha \phi^2)R + \frac{1}{2}AR^2 - \frac{1}{2}g^{\mu\nu} \partial_\mu \phi \partial_\nu \phi - V(\phi) \right]
        \end{dmath}

        ここで,$M_{\text{Pl}}$は換算プランク質量,$\alpha$はスカラー場と一般相対論の結合定数,$\phi$はインフレーションを引き起こすスカラー場であるインフラトン,$A$はStarobinsky項の修正定数である。ポテンシャルは次の通りである。

        \begin{equation}
        V(\phi) = \frac{m^2}{2}\phi^2 + \frac{\beta}{3}\phi^3 + \frac{\lambda}{4}\phi^4
        \end{equation}