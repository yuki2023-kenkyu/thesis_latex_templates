
% 付録は chapter の 1 つとして作りますが、章番号は表示しません。
% また付録の 1 つずつはアルファベットで番号付けをするのが一般的です。
\setcounter{section}{0}% section の番号をゼロにリセットする
\renewcommand{\thesection}{\Alph{section}} % 数字ではなくアルファベットで数える
\setcounter{equation}{0}% 式番号を A.1 のようにする
\renewcommand{\theequation}{\Alph{section}.\arabic{equation}}
\setcounter{figure}{0}% 図番号
\renewcommand{\thefigure}{\Alph{section}.\arabic{figure}}
\setcounter{table}{0}% 表番号
\renewcommand{\thetable}{\Alph{section}.\arabic{table}}
\lstset{
  language = Python,
  %背景色と透過度
  backgroundcolor={\color[gray]{.90}},
  %枠外に行った時の自動改行
  breaklines = true,
  %自動改行後のインデント量(デフォルトでは20[pt])
  breakindent = 10pt,
  %標準の書体
  basicstyle = \ttfamily\scriptsize,
  %コメントの書体
  commentstyle = {\itshape \color[cmyk]{1,0.4,1,0}},
  %関数名等の色の設定
  classoffset = 0,
  %キーワード(int, ifなど)の書体
  keywordstyle = {\bfseries \color[cmyk]{0,1,0,0}},
  %表示する文字の書体
  stringstyle = {\ttfamily \color[rgb]{0,0,1}},
  %枠 "t"は上に線を記載, "T"は上に二重線を記載
  %他オプション:leftline,topline,bottomline,lines,single,shadowbox
  frame = TBrl,
  %frameまでの間隔(行番号とプログラムの間)
  framesep = 5pt,
  %行番号の位置
  numbers = left,
  %行番号の間隔
  stepnumber = 1,
  %行番号の書体
  numberstyle = \tiny,
  %タブの大きさ
  tabsize = 4,
  %キャプションの場所("tb"ならば上下両方に記載)
  captionpos = t
}
