\usepackage[japanese,english]{babel}
\usepackage{./preambles/jtygm}%%%%%%%%%%%%%%%%%Font Warning対策%%%%%
\usepackage{./preambles/ads_ref}%%%%%%%%%%%%%%%%%Font Warning対策%%%%%
\usepackage[dvipdfmx]{graphicx}
\usepackage{bmpsize}
\usepackage{here}
\usepackage{comment}
\usepackage{amsmath, amssymb, amsthm, physics}
\usepackage{mathtools}% あからさまに読み込む
\usepackage{booktabs}
\usepackage[dvipdfmx,colorlinks=false]{hyperref}
\usepackage{pxjahyper}%hyperrefの不具合対策
\usepackage{titlesec}
\usepackage{wrapfig}
\usepackage{enumerate}
\usepackage{caption}
\usepackage{breqn}%数式dmath環境の使用
\usepackage{listings, ./preambles/jlisting, color}
\usepackage{cleveref}
% \usepackage{autonum}%参照していない数式には番号をつけない
\usepackage[version=4]{mhchem}%化学式
\usepackage{siunitx}%単位の表記
\usepackage[%
	backend=biber,
	style=numeric,
	sorting=none
]{biblatex}

% 参考文献ファイルの読み込み
\addbibresource{./references/reference_1.bib}

\crefformat{chapter}{第#2#1#3章}
\crefformat{section}{#2#1#3節}
\crefformat{subsection}{#2#1#3節}
\crefformat{enumi}{#2(#1)#3}
\crefrangeformat{enumi}{#3(#1)#4--#5(#2)#6}
\crefmultiformat{enumi}{#2(#1)#3}{,#2(#1)#3}{,#2(#1)#3}{,#2(#1)#3}

\numberwithin{equation}{section}%数式番号の出力形式の定義
\crefname{equation}{式}{式}% {環境名}{単数形}{複数形} \crefで引くときの表示
\crefname{dmath}{式}{式}% {環境名}{単数形}{複数形} \crefで引くときの表示
\crefname{figure}{Fig.}{Fig.}% {環境名}{単数形}{複数形} \crefで引くときの表示
\crefname{table}{Tab.}{Tab.}% {環境名}{単数形}{複数形} \crefで引くときの表示

\newcommand{\crefpairconjunction}{と}
\newcommand{\crefrangeconjunction}{から}
\newcommand{\crefmiddleconjunction}{,}
\newcommand{\creflastconjunction}{,および}

% 図のキャプションのカスタマイズ
% https://karat5i.blogspot.com/2014/10/latex.html
\captionsetup[figure]{format=plain, labelformat=simple, labelsep=space, font=footnotesize}
% 図,表番号を"<章番号>.<図番号>” ,"<章番号>.<表番号>” へ
\renewcommand{\thefigure}{\thechapter.\arabic{figure}}
\renewcommand{\thetable}{\thechapter.\arabic{table}}
%数式フラグ:「eq:」
%図フラグ:「fig:」